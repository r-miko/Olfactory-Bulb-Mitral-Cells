\documentclass[11pt]{report}
\newcommand{\userName}{Rebecca Miko}
\newcommand{\institution}{University of Hertfordshire}

%Packages
\usepackage{graphicx}
\usepackage{xr}
\usepackage{hyperref}
\usepackage{geometry}
\usepackage{booktabs}

\geometry{margin=1in}
\setlength\parindent{0pt}

% Begin document
\begin{document}

{\Huge Abstract}\\

Rebecca Miko$^1$, Christoph Metzner$^1$ and Volker Steuber$^1$

$^1$ Centre for Computer Science and Informatics Research, University of Hertfordshire, Hatfield, United Kingdom 

E-mail: rebeccamiko@herts.ac.uk


We tuned the parameters of the mitral cells by running the OB model as a function, with a range of parameters that adjust the feed-forward inhibition and the input frequency.
A tuning curve derived from the anaylysis function for firing rates and latency is plotted against $frequency$. 
The peaks are then extracted from the results to create a contour plot, which shows that there is a shift to the right of resonance as the $PGInput$ increases.
The peak frequency of the tuning curve appears to decrease as the $Excitation Factor$ increases.
Whereas, the $Inhibition Factor$ has little to no effect on the results.
After the location of resonance was found, we created a second contour plot to consider the strength of the resonance.
The second contour plot shows that the resonacne strength increases when the $Exitation Factor$ is high and the $PGInput > 0.5$.  




\end{document}\grid
